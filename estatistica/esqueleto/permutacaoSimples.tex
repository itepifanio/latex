Permutações simples são arranjamentos ordenados de objetos distintos, onde \textbf{\textit{n}} objetos são tomados \textbf{\textit{r}} a \textbf{\textit{r}}, com \textbf{\textit{r}} \textit{igual} \textbf{\textit{n}} (\textbf{\textit{r}} = \textbf{\textit{n}}). Assim, podemos dizer que uma permutação simples é dada por P(n,n), onde
\begin{center}
	$P(n,n)=n \cdot (n-1) \cdot (n-2) \cdot (n-3) \cdots 1$
\end{center}

\noindent 
que é a definição de fatorial, ou seja, o produto de todos os inteiros de \textbf{\textit{n}} até 1 \cite{niven1965mathematics}. Assim,
\begin{center}
	$P(n,n)=n!$
\end{center}

\noindent
Exemplo 1: De quantas formas diferentes o nome Djackson pode ser escrito?

Podemos dizer que essa é uma permutação de oito letras tomadas $8$ a $8$, ou seja, para o primeiro espaço temos oito letras, para o segundo sete, para o terceiro $6$ e assim por diante. Logo, temos

\begin{center}
	$P(8,8)=8!$
	
	$P(8,8)=8 \cdot (8-1) \cdot (8-2) \cdot (8-3) \cdots 1$
	
	$P(8,8)= 8 \cdot 7 \cdot 6 \cdot 5 \cdot 4 \cdot 3 \cdot 2 \cdot 1$
	
	$P(8,8)=40320$ possibilidades 
\end{center}

\noindent
Exemplo 2: Considere todos os anagramas formados por 6 letras distintas obtidas permutando-se, de todas as formas possíveis, as letras G, I, T, H, U, B. Quantos palavras é possível formar (no total) e quantas palavras se iniciam com o a letra G?

Para calcular a quantidade total de anagramas podemos realizar uma permutação P(6,6)

\begin{center}
	$P(6,6)=6!$
	
	$P(6,6)= 6 \cdot 5 \cdot 4 \cdot 3 \cdot 2 \cdot 1$
	
	$P(6,6)=720$ palavras
\end{center}

Para saber quantas iniciam com a letra G, podemos fixar a letra G na primeira das seis posições disponíveis e realizar uma permutação com as cinco letras restantes.Dessa forma, obtemos
\begin{center}
	$P(5,5)=5!$
	
	$P(5,5)=5 \cdot 4 \cdot 3 \cdot 2 \cdot 1$
	
	$P(6,6)=120$ palavras começam com G
\end{center}