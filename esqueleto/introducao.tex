Análise combinatória é o estudo da contagem de elementos finitos, analisando combinações e possibilidades. Não é uma área recente na matemática, seus estudos datam de 2000 a.c. quando matemáticos buscavam estudar os quadrados mágicos. Porém, a teoria combinatória surgiu definitivamente no século XVII, bem apresentada por três livros, um de Pascal, um de Leibniz e um outro de Athanasius Kircher.

Os termos aqui empregados como arranjos, permutações e combinações, mais utilizados por razões didáticas, só foram introduzidos na matemática depois do século XIX. Uma definição mais concreta para essa área é dada a seguir:

\begin{citacao}
	Na  análise  combinatória  estuda-se  formação,  contagem  e  propriedades  dos  
	agrupamentos que podem constituir-se, segundo determinados critérios, com os objetos 
	de   uma   coleção.   Esses   agrupamentos   distinguem-se,   fundamentalmente,   em   três   
	espécies: 
	arranjos,  permutações  e  combinações,
	e  podem  ser  formados  de  objetos  
	distintos ou repetidos \cite[p.5]{vazquez2004analise}
\end{citacao}

Esse estudo tomou como base as seguintes referências: \citeonline{niven1965mathematics}, \citeonline{dantas2013probabilidade} e \citeonline{vazquez2004analise}



