Na análise combinatória, o princípio fundamental da contagem (PFC) é um dos procedimentos utilizado para contar o número de possibilidades que uma tarefa pode ser realizada sem que seja necessário realizar essa contagem manualmente.
Neste princípio, considerando uma tarefa dividida em duas etapas, se tivermos \textbf{\textit{n}} escolhas para a primeira etapa e \textbf{\textit{m}} escolhas para a segunda etapa, então tarefa completa pode ser executada de \textbf{\textit{n.m}} maneiras, considerando que as escolhas são independentes (Dantas, 2013).

Por exemplo, considerando um homem que decide ir para Europa de avião e voltar de barco. Se há cinco linhas aéreas diferentes disponíveis para ele e sete companhias de barco, então ele poderia fazer essa viagem de 5.7 ou 35 formas diferentes.
Esse princípio pode ser extendido para escolhas além de duas etapas, para três etapas, quatro ou mais.

Formulando essa definição em termos de eventos temos que se um evento pode ocorrer de \textbf{\textit{n}} maneiras, e um segundo evento pode ocorrer independentemente do primeiro de \textbf{\textit{m}} maneiras, então os dois eventos podem ocorrer em \textbf{\textit{n.m}} maneiras (Niven, 1965).

\noindent
Exemplo 1: Quantos inteiros entre 100 e 999 possuem dígitos diferentes?

Nesse caso, não podemos apenas considerar o número de permutações dos 10 dígitos existentes (0, 1, 2, 3, 4, 5, 6, 7, 8, 9) tomados três a três, pois 052, por exemplo, não é um número entre 100 e 999. Assim, não podemos utilizar o dígito 0 para ocupar o primeiro espaço, restando nove opções para o dígito das centenas. Para o segundo espaço também temos nove opções, pois agora podemos utilizar o 0 ou qualquer um dos outros dígitos que não tenha sido utilizado ainda. De forma similar, para o último espaço teremos 8 opções.
Então, utilizando o princípio fundamental da contagem temos 

\begin{center}
	$9.9.8 = 648 \ \ inteiros \ \ entre \ \ 100 \ \ e \ \ \ \ 999 \ \ com \ \ {d\acute{\imath}gitos} \ \ diferentes$ 
\end{center}

\noindent
Exemplo 2: Dos 648 inteiros do problema anterior, quantos são ímpares?

Para o número ser ímpar ele precisa terminar em 1, 3, 5, 7 ou 9, assim, para o dígito das unidades, teremos 5 opções.
Para o dígito das centenas não podemos começar com 0, já que estes inteiros estão entre 100 e 999, logo, restam os outros oito dígitos não nulos.
Por fim, para o dígito das dezenas temos ainda 8 opções, pois agora pode ser utilizado o 0 ou um dos  outros sete dígitos não nulos. Assim, pelo PFC temos
\begin{center}
	$8.8.5 = 320 \ \ inteiros \ \ {\acute{\imath}mpares}\ \ entre \ \ 100 \ \ e \ \ \ \ 999 \ \ com \ \ {d\acute{\imath}gitos} \ \ diferentes$ 
\end{center}


