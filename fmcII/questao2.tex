Sejam 
$ f \colon A \to B $ e $ g \colon B \to A $ 
funções totais \hypertarget{h1}{(H1)}, e tais que 
$ f \circ g = Id_{B} $ 
\hypertarget{h2}{(H2)}, mas 
$ g \circ f \neq IdA $ \hypertarget{h3}{(H3)}.

(2a) Demonstre que a função f não pode ser injetiva.

(2b) Demonstre que a função g não pode ser sobrejetiva.

\begin{comment}
Como $ f \colon A \to B $ e $ g \colon B \to A $ e $ f \circ g = Id_{B} $ segue-se pelo \hyperlink{q2:lema1}{lema 1} que $ f $ é sobrejetiva e $ g $ é injetivas.
\end{comment}

\subsubsection*{2a)} 

Suponha que $ f $ é injetora. 

Seja $ a,b \in A $ tal que $ f(a) = f(b) $

Note que $ f(a), f(b) \in B $, logo $ (f \circ g)(f(a)) = Id_{A}(f(a)) $ e $ (f \circ g)(f(b)) = Id_{A}(f(b)) $

Pela definição de identidade: $ (f \circ g)(f(a)) = f(a) $ e $ (f \circ g)(f(b)) = f(b) $

Logo, $ (f \circ g)(f(a)) = f(a) = f(b) = (f \circ g)(f(b)) $

Pela definição de composição: $ f(g(f(a))) = f(a) = f(b) = f(g((f(b))) $

Logo,  $ g(f(a)) = a = b = g(f(b)) $

Note que, $ g(f(a)) = Id_{A}(a) = Id_{A}(b) = g(f(b)) $, pela definição de identidade

Entretanto, isso contradiz \hyperlink{h3}{(H3)}, portanto $ f $ não pode ser injetiva.

\subsubsection*{2b)}

 
