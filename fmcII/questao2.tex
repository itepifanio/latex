\noindent
\subsection*{
02 - Sejam 
$ f \colon A \to B $ e $ g \colon B \to A $ 
funções totais \hypertarget{h1}{(H1)}, e tais que 
$ f \circ g = Id_{B} $ 
\hypertarget{h2}{(H2)}, mas 
$ g \circ f \neq IdA $ \hypertarget{h3}{(H3)}.
}



2a) Demonstre que a função f não pode ser injetiva.

2b) Demonstre que a função g não pode ser sobrejetiva.


\subsubsection*{2a)} 

Suponha que $ f $ é injetora. 

Por (H3) existem $ (a,a') \in A $, com $ a \neq a' $ e $ (a,a') \in g \circ f $

Pela definição de composição, existe $ b \in B $ tal que $ (a,b) \in f $ e $ (b,a') \in g $

Por (H2), $ (f \circ g)(b) = Id_{B}(b)$, pela definição de identidade e composição $ f(g(b)) = b $

Note que $ g(b) =  a' $, logo $ f(a') = b $
 
Note que, $ f(a) = b $, logo $ f(a) = f(a') $

Como $ f $ é injetiva, então $ a = a' $, o que contradiz (H3), portanto $ f $ não é injetiva.

\subsubsection*{2b)}

Suponha que $ g $ é sobrejetiva.

Pela definição de sobrejetividade, existem $ (b, b') \in B $ tal que $ g(b) = b'

 
